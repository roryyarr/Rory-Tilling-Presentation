\documentclass[class=article, crop=false]{standalone}
\usepackage{tikz}
\usepackage{subcaption}
\usetikzlibrary{calc}
\usepackage{amsmath}

\begin{document}
\begin{tikzpicture}
    % Define the number of sides and the radius
    \def\sides{8}
    \def\radius{3}
    \def\smallradius{2.3}
    \def\sm{1.24}
    \def\smm{0.94}
    \def\smmm{0.51}
    \def\angle{360/\sides}
    \def\sidelength{2*\radius*sin(180/\sides)} % Side length of the first octagon
    \def\sidelength2{\radius*sin(180/\sides)} % Side length of the first octagon

    %\def\smallradius{{\sin(\angle)}}

    
    \fill[red]  (0,0) circle(3.5pt);
    
    \fill  \foreach \i in {0,...,\sides} {(\i*\angle:\radius) circle(3.5pt)};

    \fill[red]  \foreach \i in {0,...,\sides} {(\i*\angle-0.5*\angle:\smallradius) circle(3.5pt)};

    \fill[blue]  \foreach \i in {0,...,\sides} {(\i*\angle:\sm) circle(3.5pt)};

    \fill[green]  \foreach \i in {0,...,\sides} {(\i*\angle-0.5*\angle:\smm) circle(3.5pt)};

    \fill[pink]  \foreach \i in {0,...,\sides} {(\i*\angle:\smmm) circle(3.5pt)};

    
    % Draw the first octagon using a foreach loop
    %\draw[thick]
        \foreach \i in {0,1,...,\sides} {
            (\i*\angle:\radius) -- (\i*\angle+\angle:\radius)
        } -- cycle;


    % Draw the second octagon shifted by half a side length
    %\draw[thick, red]
        \foreach \i in {0,1,...,\sides} {
            (\i*\angle-0.5*\angle:\sidelength) -- (\i*\angle+0.5*\angle:\sidelength)
    } -- cycle;


    %\draw[blue] ($(5*\angle:\radius) - (6*\angle:\radius)$) -- (5*\angle:\radius){};
    \draw[blue] ($(0,0) - (4*\angle:\sm)$) --($(5*\angle:\sm) - (6*\angle:\sm) - (4*\angle:\sm)$){};
    \draw[blue] (0,0) --($(5*\angle:\sm) - (6*\angle:\sm)$){};
    \draw[blue] (1*\angle:\sm) -- (2*\angle:\sm){};
    
\end{tikzpicture}
\end{document}